% !TeX root = ../thuthesis-example.tex

% 中英文摘要和关键字

\begin{abstract}
白血病是一种常见多发且较为凶险的血液疾病,其早期发现与治疗至关重要。
目前白血病的临床诊断主要依靠医师对骨髓穿刺涂片进行形态学检查。
人工镜检存在着繁琐费时、结果主观性强等缺点,并且培养病理诊断医师要耗费大量时间。
近年来,计算机科学技术高速发展,人工智能技术被广泛应用于医学领域,其高效的计算与分析能力为疾病诊疗
带来了深刻的变革。目前诸多研究学者开始研究基于深度学习的方法来自动定位与识别血细胞,上述研究主要是通过目标检测识别网络进行迁移学习,
缺少对于骨髓血细胞特性的定制化改进。因此,我们与邃蓝智能科技 (上海) 有限公司进行合作,
针对骨髓血细胞检测与识别关键问题进行研究,提高骨髓血细胞检测识别算法精度,并编写相关软件,为医生的临床诊断提供参考依据。

首先,在数据方面,本文采用主动学习技术实现了骨髓血细胞数据标注迭代框架,并在医生的协作下快速完成了多批次骨髓血细胞
边界框与类别信息的标注。在骨髓血细胞检测方面,本文对比了多种单阶段与双阶段目标检测网络,将性能优异RetinaNet作为基线模型。
接着,对比了RetinaNet网络在仅检测与检测识别一体化任务上准确率的差异,确定了先检测再识别的骨髓血细胞处理流程。
针对密集、堆叠与黏连骨髓血细胞区域的漏检、错检等问题,本文提出了全局注意力自下而上路径聚合网络结构,IOU预测分支与
最优输运的全局最优标签分配策略,改进方法平均检测精度相比于主流目标检测网络提升了1.26\% ,达到了较为先进的性能。
在骨髓血细胞识别方面,本文对Vision Transformer网络进行改进,实现了骨髓血细胞的细粒度识别。
该方法采用基于滑动窗的图像块划分方法保留局部细节信息,提出辨识性区域选择模块,引导网络对细胞核、
细胞质等细微差异区域进行聚焦,引入了对比损失,约束骨髓血细胞的类间差异性与类内聚集性。
本文方法在慕尼黑骨髓血细胞数据集上的平均分类准确率为 91.96\%,相比于基线模型提升了0.82\%。
最后,本文基于B/S架构设计并开发了骨髓血细胞检测与识别软件,将上述骨髓血细胞检测识别算法进行落地部署,
软件可以自动完成骨髓血细胞的定位与分类计数,用于辅助医生临床诊断。




  % 关键词用“英文逗号”分隔,输出时会自动处理为正确的分隔符
  \thusetup{
    keywords = {医学图像分析, 骨髓血细胞, 深度学习, 检测识别},
  }
\end{abstract}

\begin{abstract*}
Leukemia is a common, multiple and dangerous blood disease, whose early diagnosis and treatment are
crucial to prolong the survival time and improve the quality of patients' live. 
At present, the clinical diagnosis of leukemia mainly relies on the morphological examination of bone marrow aspiration smear by physicians.
Manual microscopic examination is very cumbersome and time-consuming and the results are highly subjective. In addition, it takes a lot of time to train experienced pathologists.
The computer science and technology has ballooned in recent years. Artificial intelligence technology has been widely used in the medical field. Its efficient computing and analysis capabilities 
have brought profound changes to the diagnosis and treatment of diseases. 
At present, many researchers have begun to study methods based on deep learning to automatically locate and identify blood cells. 
The above researches are mainly based on general-purpose target detection and recognition networks 
and lack improvements for the characteristics of bone marrow blood cells.
Therefore, we cooperate with Deep Voxel Intelligent Technology to research the key issues of bone marrow blood cell 
detection and recognition. We aims to improve the accuracy of bone marrow blood cell detection and recognition and make software to provide reference for doctors' clinical diagnosis.

First, in terms of dataset, we used active learning technology to realize the iterative framework for bone marrow blood cell data 
annotations and quickly completed the annotations of bone marrow blood cell dataset with the cooperation of doctors.
In terms of bone marrow blood cell detection, we compared a variety of one-stage and two-stage target detection networks. 
We finally used RetinaNet as the baseline model. We compared the accuracy of the RetinaNet network between the task of only detection and the integration task of detection and recognition
and determined the processing flow of detection first and then recognition.
Aiming at the missed detection and wrong detection of stacked and cohesive bone marrow blood cell regions,
we propose a global attention bottom-up path aggregation structure, IOU prediction branches and optimal transport label assignment strategy.
Compared with the mainstream target detection network, our method
improved the average detection accuracy by more than one percent and achieved state of the art.
In terms of bone marrow blood cell recognition, we realized the fine-grained recognition of bone marrow blood cells based on the Vision Transformer network.
In this method, we adopt a sliding window-based image patch division method to preserve local detail information. 
In addition, we propose a discriminative region selection module to guide the network to focus on nuanced regions such as nucleus and cytoplasm. 
A contrast loss is introduced to constrain the inter-class diversity and intra-class aggregation of bone marrow blood cells.
The average classification accuracy of this method on the Munich bone marrow blood cell dataset is 91.96 percent, 
which is 0.82 percent higher than the baseline model.
Finally, we designed and developed the bone marrow blood cell detection and recognition software based on the B/S architecture.
We deployed the above bone marrow blood cell detection and recognition algorithm in this software. 
The software can automatically complete the positioning, 
classification and counting of bone marrow blood cells to assist doctors in clinical diagnosis.

  \thusetup{
    keywords* = {Medical Image Analysis, Bone Marrow Blood Cells, Deep Learning, Detection and Recognition},
  }
\end{abstract*}
