% !TeX root = ../thuthesis-example.tex

\chapter{绪论}
\section{研究背景与意义}
骨髓是人体最主要的造血器官,其存在于人体骨骼内部的空腔中,约占体重的3.5~5.9\%。
作为人体的造血组织,骨髓中包含了多种不同发育阶段的血细胞,
这些血细胞按照形态与功能可以划分为粒细胞系、红细胞系、淋巴细胞系、单核细胞系与浆细胞系。
骨髓血细胞成熟后会通过密质骨中的连通管等进入人体外周血,参与人体循环系统的血液循环,保证机体新陈代谢的进行。

血细胞的质与量出现异常通常与某种血液疾病密切相关。白血病\cite{黄治虎2009我国白血病流行病学调查的现状和对策}是一种常见多发的血液疾病,主要表现为细胞异常克隆增生
导致的骨髓造血功能异常。白血病属于人体造血系统的恶性肿瘤,
在所有恶性肿瘤中占比约5\%,是我国重点防治的十大恶性肿瘤之一。
白血病患者临床症状为贫血、出血、发热、乏力等,其致死率较高,早期发现与治疗对延长患者生存时间、改善患者生活质量至关重要。

骨髓血细胞形态学检查是精确诊断白血病类型的关键手段之一\cite{heimpel1979conventional}。
目前,大型医院的骨髓血检查主要依靠病理学医师对显微设备采集的血细胞图像进行观察,并人工分类计数。
检测流程首先需制备骨髓涂片并使用瑞特与吉氏混合液进行染色。接着,使用低倍显微镜判断骨髓增生程度、
观察是否存在异常形态的特殊细胞。在低倍镜观察完全片后,再使用油镜从骨髓涂片中部向尾部移动,记录约500个有核细胞中各类
血细胞的数量。目前人工镜检存在以下不足,人工分类计数过程非常枯燥且繁琐费时,通常需要数个工作日后才能出具诊断报告,
不能满足快速临床诊疗的需求。对医师的专业技术要求较高,培养精通细胞病理诊断的医师要周期长,年轻医师从事人工镜检的意愿低。
诊断结果依赖于医师的专业知识与经验,存在较强的主观性,诊断的规范性与一致性较差。

在过去的20年间,计算机科学技术高速发展,医疗硬件设备的不断提升,医学领域积累了大量的医疗诊断数据,
人工智能(AI,Artificial Intelligence)技术被广泛应用于医学领域\cite{2018Data}。目前AI已经在医疗机器人、药物研发、智能问诊、智能影像识别等领域
进行落地与应用。AI高效的计算与分析能力极大提升了医生的工作效率,为疾病检测与诊疗带来了深刻的变革。
在血细胞图像智能诊疗方面,诸多研究学者采用深度学习的方法来自动定位与识别血细胞,实现了快速筛选和分类计数。这项技术使得细胞形态学诊断变的自动化、标准化与智能化,将医生从繁重的细胞病理工作中解放出来,具有非常重要的临床辅助诊断的意义。

目前骨髓血细胞自动化检测与识别技术已取得了长足的进步,但仍然面临着诸多挑战。在血细胞检测方面,涂片背景复杂干扰较多、细胞间相互黏连与重叠会影响检测结果的精确度。
在血细胞识别方面,骨髓血细胞种类非常多,存在各类细胞样本数量不均衡、细胞类内差异大、相邻发育阶段细胞类间差异小等问题。
因此,基于深度学习的血细胞自动检测与识别方法仍有巨大的提升空间。本文针对骨髓血细胞检测与识别关键问题进行研究,并编写相关软件,
为医生的临床诊断提供参考依据,具有非常重要的临床意义与广阔的应用前景。



\section{研究现状与进展}
\subsection{骨髓血细胞图像检测现状}
在血细胞涂片图像处理的过程中,包含血细胞区域的提取至关重要,这是后续识别过程的基础,分割结果的精确性对后续任务有很大的影响。
如何精准的从血细胞涂片中分割出各类细胞的边界,定位包含血细胞的区域是医学图像处理的重要研究方向之一。
近几十年来,国内外学者对此进行了深入的研究,并提出了多种解决方案,大致可以划分为以下四类。
\subsection{骨髓血细胞图像识别现状}
\section{本文内容及章节安排}

