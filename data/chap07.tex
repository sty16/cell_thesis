\chapter{总结与展望}
\section{全文总结}
过去的十年中,计算机科学技术飞速发展,人工智能广泛赋能于医学行业,为药物研发、疾病诊疗等领域带来了深刻的变革。
如今,各大医院已经积累了海量患者数据,为基于数据驱动的深度神经网络提供了数据支撑。白血病是一种常见多发的血液疾病,
目前该疾病的诊断主要依靠医生在显微镜下对骨髓血细胞进行形态学检查。全自动化的骨髓血细胞分类计数,可以辅助医生临床诊断,具有广阔的应用前景。

复杂背景下的骨髓血细胞检测与识别是实现全自动化骨髓血细胞形态学检查的重要研究课题,本文以邃蓝智能科技(上海)有限公司提供的骨髓血细胞样本为例,深入分析了目前骨髓血细胞检测与识别研究
面临的技术难点,并研究了基于深度学习的骨髓血细胞检测识别算法,最后将相关算法进行落地,设计并实现了骨髓血细胞检测识别软件。
本文的主要工作包含了以下的四个方面:

1)提出主动学习的数据标注迭代框架、先检测再识别的算法策略。

本文使用数据来源于实践基地邃蓝智能科技有限公司合作医院提供的脱敏数据。原始数据无标签,\
首先在医生协作下完成数据集血细胞边界框与类别信息的标注。大量的数据标注耗时费力,本文采用主动学习技术,
标注少量数据训练模型,对未标注的数据根据网络输出的熵值降序排序,去发掘数据集中高信息量的样本,提高标注的效率与精度,
降低标注成本。在骨髓血细胞的检测方面,本文对比了多种单阶段与双阶段目标检测网络的检测精度、计算量与速度等性能,
并确定将RetinaNet作为骨髓血细胞检测的基线模型。本文探索了RetinaNet网络在仅检测与检测识别一体化任务上特征提取与识别准确率上
的差异,实验结果表明检测识别一体化任务的输出类别置信度较低,平均识别正确率只有 58.1\%,易发生漏检与误检。
仅检测任务的平均精度达到了94.2\%,检测的准确率与召回率都有大幅提升。本文认为两种任务共享骨干网络提取的特征,
导致骨干网络难以对精细分类特征进行学习,因此确定了先检测再识别的骨髓血细胞处理流程,即先使用检测网络定位到血细胞的位置,
然后剪切为血细胞切片,再输入到识别网络中进行分类的流程。

2)提出了基于改进RetinaNet的骨髓血细胞检测算法

确定了先检测再识别的流程后,骨髓血细胞检测部分更加关注血细胞定位精度。为了解决密集、堆叠与黏连血细胞区域的漏检、错检等问题,
本文提出了一种基于改进RetinaNet的骨髓血细胞检测网络。该网络采用了基于全局注意力的自下而上的路径聚合模块,
通过特征直连缩短了底层到顶层特征之间的信息传递路径,使得浅层的纹理等高分辨定位信息可以更有效的传递到顶层,提升网络定位特征的表达能力。
网络引入了IOU预测分支,对每个锚框与可能对应真实框交并比进行预测,在推理阶段将检测框的定位质量也纳入到非极大值抑制的考量当中。
此外,本文探究了不同标签分配策略对检测性能的影响,提出基于最优输运的标签策略用于密集区域的血细胞的标签分配,避免了模糊分配样本的出现,
提高网络对血细胞的召回能力。在骨髓血细胞数据集上的实验结果表明,本文提出的改进方法在检测精度(mAP)上相比于主流的目标检测网络提升了1\%以上,
达到了较为先进的性能。网络单张图像的平均检测时间约为37ms,可以用于实时全自动化骨髓血细胞检测。

3)提出了基于改进Vision Transformer的骨髓血细胞识别算法

针对骨髓血细胞种类繁多,相邻发育阶段的血细胞在形态上非常类似,骨髓血细胞子类之间差异较小等识别难点,
本文将性能优异的 Vision Transformer 作为基线模型,提出了重叠图像块划分方法、辨识性区域选择模块与对比loss对基线模型进行改进。
其中重叠图像块划分方法可以更好的保留图像的局部信息、避免破坏图像的局部结构。辨识性区域选择模块采用压缩激发结构学习所有编码层的注意力图权重,
并将每个注意力头最大权重对应的隐含特征用于分类,该模块让网络关注到不同类别之间的细微差异部分,同时舍弃大量区分度较低的背景、超类共同特征区域\cite{SWGC202206005},
从而提升网络的细粒度特征表达能力。为了进一步增加分类特征的类内一致性与类间差异性,本文引入了对比损失,对特征的间隔进行约束,
使得不同标签特征相似度最小,相同标签的特征相似度最大。实验结果表明,本文方法在慕尼黑骨髓血细胞数据集上的平均分类准确率为91.96\%,
与卷积神经网络相比识别准确率提升了1.5\% $\sim$ 3.0\%,相比于基线Vision Transformer模型识别准确率提升了0.74\%,有望为医生临床诊断提供参
考依据,具有潜在的临床应用前景。

4)将上述算法落地,研究并开发了骨髓血细胞检测与识别软件

本文基于B/S架构设计并开发了骨髓血细胞检测与识别软件。前端使用 Vue 框架与 Element UI 组件库。
后端使用的框架 Django2.4,深度学习模型部署工具 ONNX,数据库为 Mysql 8.0。软件实现了用户模块、
骨髓血细胞检测模块、骨髓血细胞识别模块与患者数据管理模块。通过该软件,医生可以将患者骨髓血细胞图像数据一键上传,
软件自动完成血细胞的定位、分类计数。最终根据 FAB分类标准给出病情的诊断。
医生可以查询相关患者的单张血细胞图像分类结果与整体血细胞分布的柱状图,并对血细胞类别进行校对与修改。
血细胞数据均落入到云端的数据库中,通过数据的不断积累,未来可以进一步提升模型的检测与识别性能。

\section{工作展望}

未来工作展望,骨髓血细胞自动化检测识别技术集成了计算机视觉、深度学习、目标检测、目标识别等技术,
其研究对于实现全自动的骨髓血细胞形态学分析具有重要的价值。本文实现了骨髓血细胞的检测与识别算法,
性能相较于基线模型有一定幅度的提升,但在实际应用中,骨髓血细胞成分非常复杂,细胞种类繁多、背景多变,未来仍需在以下三个方面展开研究:

(1)持续收集骨髓血细胞数据集,并对血细胞图像类别进行人工标注。目前单核细胞、早幼粒细胞等由于数据较少,
识别效果较差,后续需要特别进行扩充。目前医院临床诊断报告中包含约30余类血细胞的计数信息,
需要继续扩充目前训练集中不存在的骨髓血细胞类别,使模型可以达到进行临床诊断的标准。

(2)尽管Vision Transformer实现了非常高的骨髓血细胞分类准确率,但是模型参数量为93MB,运算量为19.44GFLOPs,
对部署的硬件环境有较高的要求。未来需要进一步精简其结构,提高模型的识别速度。可以采用神经网络结构搜索方式,
设计超网络结构,采用基于可微分的搜索策略如DARTS\cite{liu2018darts}、GDAS\cite{dong2019searching}等,对网络中不重要的分支与结构进行裁剪,
找到更加精准的模型速率与准确率的平衡\cite{SWGC202206005}。

(3)有标签骨髓血细胞数据获取的成本极其昂贵,目前医院有海量的无标签骨髓血细胞数据,未来研究应该聚焦
与如何将利用大量的无标签数据来提升模型的性能。例如利用无监督学习\cite{berry2019supervised}通过大量的无标签数据进行预大模型的预训练,然后基于
少量的有标签数据进行下游任务的迁移。或者采用伪标签的半监督学习\cite{ouali2020overview}方式给无标签的数据打上标签信息,
然后先利用这些伪标签进行训练,再进行微调。半监督/无监督学习可以降低标注数据的需求、通过使用未标记数据来学习更多的特征,
使得模型对噪声和异常值更加鲁棒,泛化性能更强。