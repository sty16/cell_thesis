% !TeX root = ../thuthesis-example.tex

\begin{acknowledgements}
2016年我怀揣着对未来的美好憧憬踏入了清华大学,转眼间已经在清华学习生活了七年。
在这七年的时光中,我收获了知识与成长,开阔了视野与认知。在漫漫求索的路途中,遇到诸多艰难险阻,
是老师、同学、朋友与家人的帮助与陪伴让我倍感温暖并激励我不断前行。值此毕业之际,向他们表达最诚挚的感谢。

首先,感谢我的导师杨健教授,在本科时期,他是我的班主任,在学业方面给予我很多指导,让我明确
的学习的目标与方向。在大三时,非常荣幸保研到了杨健教授师门下,杨健教授在确定科研方向、文献阅读与代码编写等方面
对我进行悉心指导,是我科研道路上的引路人。在研究生三年时光中,杨健教授为我提供了企业、医院与研究所等实践机会,
指导我参与并完成了多项科研项目,让我的认知与技术能力得到了极大提升。在生活方面,杨健教授亲切随和、亦师亦友,
对我们的关心无微不至。杨健教授渊博的学识、精益求精的治学态度、严谨认真的科研作风,为后辈树立了伟岸的学习榜样。

然后,感谢实践基地邃蓝智能科技公司首席曾亮老师,感谢他与公司对本项目的大力支持。
曾亮老师在骨髓血细胞检测与识别算法研究中为我提供了专业的理论指导,让我顺利的完成了硕士课题。

接着,感谢实验室的师兄师姐师弟师妹们,与你们在实验室的相处,让我的生活更加丰富多彩,平时的研究交流也让我受益匪浅。
感谢王洪淼师兄在深度学习与机器学习方向给予的指导;感谢何耀民师兄在读硕期间的指导与帮助,
疫情期间我们被封控在宿舍内,相互扶持鼓励,共同度过了最艰难的一段时光;
感谢朱庆涛师兄在数据集处理与文献调研方面给予的帮助与建议。祝大家未来前程似锦,江湖再见!

最后,感谢我的父母,感谢他们多年来的辛勤付出与培养,他们在生活中无微不至的关心与照顾是我勇敢前行的坚实后盾。







\end{acknowledgements}
