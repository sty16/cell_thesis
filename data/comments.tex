% !TeX root = ../thuthesis-example.tex

\begin{comments}
% \begin{comments}[name = {指导小组评语}]
% \begin{comments}[name = {Comments from Thesis Supervisor}]
% \begin{comments}[name = {Comments from Thesis Supervision Committee}]
论文研究基于深度学习的骨髓血细胞检测与识别技术,选题具有重要的临床意义与广阔的应用前景。
作者阅读了大量相关文献,对国内外研究现状概述准确,并在此基础上进行深入研究,其主要工作与创新为:

(1) 针对骨髓血细胞检测问题,提出了一种改进的RetinaNet网络,采用全局注意力自下而上路径聚合网络结构,IOU预测分支与最优输运的全局最优标签分配策略,提升了骨髓血细胞检测的精度与性能。 

(2) 在骨髓血细胞识别方面,基于骨髓血细胞特性对Vision Tranformer网络进行改进,提出了辨识性区域选择模块、滑动窗的图像块划分方式并引入对比损失,进一步增强骨髓血细胞识别的准确率与泛化性能。

(3) 设计并实现了骨髓血细胞检测识别软件,集成了上述创新算法,可用于临床辅助诊断。 

论文工作表明作者掌握了本学科坚实的基础理论与系统的专门知识,具备独立从事本科研工作的能力。论文结构清晰完整、逻辑严密、写作规范,达到了硕士学位论文的要求。

\end{comments}
